\documentclass{ctexart}
\usepackage[utf8]{inputenc}
\usepackage{geometry}
\usepackage{graphicx}
\usepackage{amsmath}
\usepackage{amssymb}
\usepackage{enumitem}
\usepackage{xcolor}
\usepackage{listings}
\usepackage{threeparttable}
\usepackage{multirow}
\usepackage{longtable}
\geometry{a4paper,left=2cm,right=2cm,top=1.5cm,bottom=1cm}
\newfontfamily\courier{Consolas}
\lstset{linewidth=1.1\textwidth,
		numbers=left, 
		basicstyle=\small\courier,
		numberstyle=\tiny\courier,  
		keywordstyle=\color{blue}\courier,
		%identifierstyle=\bf,
		commentstyle=\it\color[cmyk]{1,0,1,0}\courier,
		stringstyle=\it\color[RGB]{128,0,0}\courier,
		%framexleftmargin=10mm,
		frame=single,
		backgroundcolor=\color[RGB]{245,245,244},
		%escapeinside=``,
		breaklines,
		extendedchars=false,
		xleftmargin=2em,xrightmargin=2em, aboveskip=1em,
		tabsize=4,
		showspaces=false
		basicstyle=\small\courier
	   }
\newcommand{\subsubsubsection}[1]{\paragraph{#1}\mbox{}\\}
\setcounter{secnumdepth}{4}
\setcounter{tocdepth}{4}

\newtheorem{Thm}{定理}
\newtheorem{Def}{定义}
\newtheorem{Lem}{引理}
\newtheorem{Cor}{推论}
\newtheorem{Rmk}{注记}

\title{论文翻译:使用参数伪流形由三角网格构造光滑曲面的新方法}
\date{2023年11月30日}

\begin{document}

\maketitle

\begin{abstract}
	为了用光滑曲面拟合具有任意复杂拓扑的三角网格结构,我们提出了一种新的基于流形的模型。
	我们的模型结合了前人设计的模型的大多数最佳特征,从而填补了它们的不足。
	我们还构造了一个理论框架,证明了模型的正确性。
	最后,我们通过几个例子证明了基于流形的模型的有效性。
\end{abstract}

\section{引入}
	将具有保持拓扑性质和连续性的曲面拟合到具有任意复杂拓扑的多边形网格顶点的问题一直是一个重要的研究课题。
	这个问题的主要困难在于,具有任意拓扑的网格通常不能在单个矩形区域上参数化。
	现有的计算方法一般沿着多项式参数曲面的边界黏合在一起,
	然而确保沿边界的连续性已被证明是一个难题,
	因此大多数只能保证低阶($C^2$及以下)连续性,
	即使存在能构造高阶连续拟合的模型,也大多是复杂和难以实现的;
	细分曲面是另一种被广泛研究的方法,它直观易于实现。
	但很少有基于这种方法达到 $C^2$ 以上的构造,且这种方法很难保证奇点附近的连续性。
	此外,对于 $k\geq 2$,生成 $C^k$ 曲面的细分方案不能像现有的 $C^1$ 细分方案那样简单。

	不同于上述两种方法,由 Grimm 和 Hughes 开创的基于流形的模型
	已被证明可以相对容易地对三角形和四边形网格拟合 $C^k$ 连续参数曲面,包括 $k=\infty$。
	流形天然地具有任意光滑性,且其微分结构为我们提供了在曲面上求解方程的天然条件。
	基于流形的模型还同时具有样条曲线、曲面的一些最重要的特性,例如对曲线形状和基函数的局部控制。

	遗憾的是,现有的基于流形的模型都存在一些缺点,例如:
	利用流形结构的定义过于复杂;
	产生具有奇点的曲面,导致连续性阶数降低;
	得到的曲面不是多项式参数(例如使用了指数函数)。
	另一方面,某些构造更简单的模型和任意平滑的方法不能建立处理三角形网格的完整框架。

	本文贡献:
	\begin{enumerate}
		\item 我们引入了一种基于流形的模型,将任意光滑($C^\infty$连续)的曲面拟合到三角形网格上。
			我们的模型结合了前人设计的模型的大多数最佳特征:
			模型紧凑、简单;
			不包含奇异点;
			能对四边形网格做拟合;
			定义的曲面几何结构灵活。
		\item 我们还简要描述了一个理论框架,为我们基于流形的构造的正确性提供了证明。
	\end{enumerate}

\section{理论基础}
	简单来说,流形是局部类似于欧氏空间的空间,我们可以在其上进行类似欧氏空间中的微积分运算。
	流形传统的处理方式是:每个流形 $\mathcal{M}$ 都配备了一个坐标图册,即一个坐标卡的集合 $\{(U_i,\varphi_i)\}_{i\in I}$。
	坐标卡定义为有序对 $(U_i,\varphi_i)$,其中
	$U_i\subset \mathcal{M}$,
	$\varphi_i:U_i\rightarrow \varphi(U_i)\subset \mathbb{R}^n$。
	要求:$\{U_i\}_{i\in I}$ 形成 $\mathcal{M}$ 的一个开覆盖;
	$\{\varphi_i\}_{i\in I}$ 要满足相容性条件,
	即对任意 $\varphi_1,\varphi_2\in \{\varphi_i\}_{i\in I}$,
	转换函数 $\varphi_{12}=\varphi_2\circ \varphi_1^{-1}$ 和 $\varphi_{21}=\varphi_1\circ \varphi_2^{-1}$ 均为同胚。

	传统的处理方式首先假设了流形的存在,我们不能通过这种方式来构造流形。
	好在我们可以用一组参数化曲面和一组粘合映射以构造性的方式定义 $\mathcal{M}$。

	\begin{Def}\label{PGM-def}
		设 $n$ 为正整数,$k\geq 1$ 为正整数或 $k=\infty$,定义

		\begin{equation}
			\mathcal{G}=((\Omega_i)_{i\in I},(\Omega_{ij})_{(ij)\in I\times I},(\varphi_{ji})_{(i,j)\in K})
		\end{equation}

		$\mathcal{G}$ 满足下面的性质,$I,K$ 是可数指标集,$I$ 非空。

		\begin{enumerate}
			\item
				对任意 $i\in i$,集合 $\Omega_i$ 为 $\mathbb{R}^n$ 上的非空开集(称为参数区域),
				$\Omega_i$ 两两不交。
			\item
				对任意 $(i,j)\in I\times I$,集合 $\Omega_{ij}$ 是 $\Omega_i$ 的开子集。
				特别地,$\Omega_{ii}=\Omega_i$,$\Omega_{ji}\neq \emptyset$ 当且仅当 $\Omega_{ij}\neq \emptyset$。
				称非空集 $\Omega_{ij},i\neq j$ 为粘连区域。
			\item
				如果令 $K=\{(i,j)\in I\times I\|\Omega_{ij}\neq \emptyset\}$,则
				对任意的 $(i,j)\in K$,$\varphi_{ij}:\Omega_{ij}\rightarrow \Omega_{ji}$ 是 $C^k$ 同胚,且满足下列条件:
				\begin{enumerate}
					\item 对任意 $i$,$\varphi_{ii}=\text{id}_{\Omega_i}$。
					\item 对任意 $(i,j)\in K$,$\varphi_{ij}=\varphi_{ji}^{-1}$。
					\item
						对所有 $i,j,k$,若 $\Omega_{ji}\cap \Omega_{jk}\neq \emptyset$,则
						$\varphi_{ji}^{-1}(\Omega_{ji}\cap \Omega{jk})\subset \Omega_{ik}$ 且
						对任意 $x\in \varphi_{ji}^{-1}(\Omega_{ji}\cap \Omega_{jk})$
						均有 $\varphi_{ki}(x)=\varphi_{kj}\circ\varphi_{ji}(x)$ 
				\end{enumerate}
			\item
				对任意的 $(i,j)\in K,i\neq j$,任意 $x\in \partial(\Omega_{ij})\cap \Omega_i,y\in \partial(\Omega_{ji})\cap \Omega_j$,
				存在开球 $V_x,V_y$ 使得 $\varphi_{ji}(V_x\cap \Omega_{ij})\cap (V_y\cap \Omega_{ji})=\emptyset$。
		\end{enumerate}
	\end{Def}

	传统的定义和基于粘合集的定义中某些概念有直接的对应关系:

	\begin{itemize}
		\item 每个参数区域都是 $\Omega_i$ 都为一个坐标卡在 $\mathbb{R}^n$ 中的像 $\varphi_i(U_i)$。
		\item 每个粘连区域 $\Omega_{ij}\subset \Omega_i$ 都为坐标卡的重叠区域在 $\mathbb{R}^n$ 中的像 $\varphi_i(U_i\cap U_j)$。
		\item 每个转换函数 $\varphi_{ij}:\Omega_{ij}\rightarrow \Omega_{ji}$ 都为坐标卡之间的转换函数。
	\end{itemize}

	上面的定义 3(c) 称为轮换条件,它在下面的定理 1 的证明中非常重要。

	依照微分流形的研究惯例,我们假设流形是 Hausdorff 的。

	\begin{Thm}\label{PGM-thm}
		对任意粘合集 $\mathcal{G} = ((\Omega_i)_{i\in I},(\Omega_{ij})_{(i,j)\in I\times I},(\varphi_{ji})_{(i,j)\in K})$,
		存在一个 $n$ 维 $C^k$ 流形 $M_\mathcal{G}$,其转换函数为全体 $\varphi_{ji}$。
	\end{Thm}

	遗憾的是,定理 \ref{PGM-thm} 的证明是非构造性的,即我们无法直接得到由粘合映射确定的流形的集合信息。
	它甚至可能不可定向或不紧致。
	但我们要解决的实际问题是要对一个三角形网格近似一个紧流形。
	事实证明,我们总可以从任意给定的粘合集定义“参数伪流形”。当某些条件成立时,这个伪流形在 $\mathbb{R}^3$ 中的图像为曲面。

	\begin{Def}\label{PPM-def}
		设 $n>d\geq 1$ 为整数,$k\geq 1$ 为整数或 $k=\infty$,
		$\mathbb{R}^n$ 中 $d$ 维 $C^k$ 参数伪流形定义如下:
		\begin{equation}
			\mathcal{M}=(\mathcal{G},(\theta_i)_{i\in I}).
		\end{equation}
		$\mathcal{G}$ 见定义 \ref{PGM-def},$\theta_i:\Omega_i\rightarrow \mathbb{R}^n$ 称为参数化,是 $C^k$ 函数,满足
		对任意的 $(i,j)\in K$ 均有 $\theta_i=\theta_j\circ \varphi_{ji}$。

		称 $\mathcal{M}=\cup_{i\in I}\theta_i(\Omega_i) \subset \mathbb{R}^n$ 为参数伪流形的像。

		特别地,当 $d=2,n=3$ 时,我们称 $\mathcal{M}$ 为参数伪曲面(PPS)。
		如果我们要求 $\theta_i$ 是双射且进一步满足
		对任意 $(i,j)\in K$,$\theta_i(\Omega_i)\cap \theta_j(\Omega_j) = \theta_i(\Omega_{ij}) = \theta_j(\Omega_{ji})$,
		且对任意 $(i,j)\notin K$, $\theta_i(\Omega_i)\cap \theta_j(\Omega_j) = \emptyset$,
		则 $\mathcal{M}$ 一定是 $\mathbb{R}^3$ 中的一张曲面。
	\end{Def}

\section{参数伪曲面(PPS)的构造}
	我们的目标是构造一个曲面 $S\subset \mathbb{R}^3$ 拟合给定的三角网格 $\mathcal{T}$
	且保持它的基空间 $|\mathcal{T}|$ ($\mathcal{T}$ 的顶点、边和三角面之并)的拓扑性质。
	我们还假设 $|\mathcal{T}|$ 是 $\mathbb{R}^3$ 中具有空边界的曲面。
	最后,为构造 $S$,我们的构造定义了 PPS 的一组粘合集和一组参数化函数。

	其中,粘合集
	$\mathcal{G} = ((\Omega_i)_{i\in I},(\Omega_{ij})_{(i,j)\in I\times I},(\varphi_{ji})_{(i,j)\in K})$
	通过 $\mathcal{T}$ 的元素定义;
	参数化 $\theta_i:\Omega_i\rightarrow \theta(\Omega_i)\subset \mathbb{R}^3,i\in I$
	通过 $|\mathcal{T}|$ 定义。
	下面我们给出 $\mathcal{G}$ 和 $(\theta_i)_{i\in I}$ 的具体构造方法。

	\subsection{粘合集 $\mathcal{G}$ 的构造}
		以下记 $I=\{u|u \in V(\mathcal{T})\}$。
		为构造粘合集 $\mathcal{G}$,我们需要定义它的参数区域、粘连区域和转换函数。
		它们通过两个概念定义:一个\textbf{P-多边形}和它的\textbf{规范三角剖分},以及一个复合双射函数。
		在我们描述这些概念之前,我们先对一些符号进行说明。

		\begin{Rmk}
			我们接下来要定义的记号都与 $\mathcal{T}$ 中的顶点(记为 $u$)或与边(记为 $[u,v]$)有关。
			因此,我们使用下标 $u$ 表示与 $u$ 有关的元素;
			使用下表 $(u,v),(v,u),uv,vu$ 表示与 $[u,v]$ 有关的元素。
		\end{Rmk}

		\begin{Def}\label{PD-def}
			对所有 $u\in I$,定义参数区域 $\Omega_u$ 为
			\begin{equation}
				\Omega_u=\{(x,y)\in \mathbb{R}^2|x^2+y^2<\cos^2(\dfrac{\pi}{m_u})\}.
			\end{equation}
			其中, $m_u$ 是 $u$ 的权值。
			对任意两点 $u,v\in I$,我们假设 $\Omega_u$ 和 $\Omega_v$ 在不相交的两个 $\mathbb{R}^2$ 平面的副本中,
			因此 $\Omega_u\cap \Omega_v=\emptyset$。
			且定义 \ref{PGM-def} 中的条件 1 成立。
		\end{Def}
		
		为建立粘合区域和转换函数,我们定义了\textbf{P-多边形}及其\textbf{规范三角剖分}的概念,
		以及由两个解析的旋转函数和双反射函数复合而成的双射。
		对 $\mathcal{T}$ 的每个顶点 $u$,与 $u$ 对应的多边形 $P_u$ 的顶点为
		$u'_i=(\cos(\dfrac{2\pi \cdot i}{m_u}),\sin(\dfrac{2\pi \cdot i}{m_u}))$,
		其中 $i=0,1,\dots,m_u-1$,$m_u$。即 $P_u$ 是正 $m_u$ 边形。
		我们假设 $P_u$ 在包含 $\Omega_u$ 的 $\mathbb{R}^2$ 副本内部。
		所以,$\Omega_u$ 是内接在 $P_u$ 中的圆 $C_u$ 的内部。即 $\Omega_u = \text{int}(C_u)$。

		我们可以通过添加 $m_u$ 条对角线和顶点 $u'=(0,0)$ 将多边形 $P_u$ 剖分为 $m_u$ 个三角形。
		每条对角线连接 $u'$ 和 $P_u$ 的一个顶点 $u'_i$。
		这样得到的三角剖分记作 $T_u$,称为 $P_u$ 的规范三角剖分。
		记 $T_u$ 的顶点集合为 $V(T_u)$,
		令 $\mathcal{N}(u,\mathcal{T})$ 为 $u$ 及 $\mathcal{T}$ 中与 $u$ 相连的点之并。
		则可定义双射 $s_u:\mathcal{N}(u,\mathcal{T})\rightarrow V(T_u)$,满足 $s_u(u)=u'$
		且 $[u,u_i,u_{i+1}]$ 是 $\mathcal{T}$ 中的三角形,
		当且仅当 $[s_u(u)=u',s_u(u_i),s_u(u_{i+1})]$ 是 $T_u$ 中的一个三角形(这里的 $+1$ 是模 $m_u$ 意义下的)。
		我们可以将这个双射 $s_u$ 延拓到三角形之间的映射。
		事实上,若 $\sigma = [u,u_i,u_{i+1}]$ 是 $\mathcal{T}$ 中的一个三角形,
		则 $s_u(\sigma) = [u',s_u(u),s_u(u_{i+1})]$ 是它在 $T_u$ 中相应的三角形。
		后文中除非我们显式表明,我们都记 $s_u(v)=v',v\in \mathcal{N}(u,\mathcal{T})$。

		对 $u\in I,p\in \mathcal{R}^2-\{0\}$,令
		\begin{equation}
			g_u:\mathbb{R}^2-\{0\}\rightarrow \mathbb{R}^2-\{0\},g_u(p) = \Pi^{-1}\circ f_u\circ \Pi(p).
		\end{equation}
		其中 $\Pi:\mathbb{R}^2\rightarrow (-\pi,\pi]\times \mathbb{R}_+$ 是极坐标变换,
		$f_u:(-\pi,\pi]\times \mathbb{R}_+\rightarrow (-\pi,\pi]\times \mathbb{R}_+$ 为
		\begin{equation}
			f_u(\theta,r) = (\dfrac{m_u}6 \cdot \theta, \dfrac{\cos(\frac{\pi}6)}{\cos(\frac{\pi}{m_u})}\cdot r).
		\end{equation}
		其中,$(\theta,r)=\Pi(p)$ 是 $p$ 的极坐标。
		函数 $g_u$ 直观的解释如下:
		它将 $C_u$ 的扇形区域 $A$ 的内部映射到半径为 $\cos(\dfrac{\pi}6)$、中心在 $(0,0)$ 的圆中的扇形区域 $B$ 中,
		$A$ 包含 $(0,0)$ 和所有极坐标在 $[-\dfrac{2\pi}{m_u},\dfrac{2\pi}{m_u}]\times (0,\cos(\dfrac{\pi}{m_u}))$ 内的点。
		$B$ 包含 $(0,0)$ 和所有极坐标在 $(\beta,s)\in [-\dfrac{-\pi}3,\dfrac{\pi}3]\times (0,\cos(\dfrac{\pi}6))$ 内的点。
		我们称 $B$ 是规范扇形。

		注意到 $g_u$ 是一个双射。其逆 $g_u^{-1}$ 为
		\begin{equation}
			g_u(p) = \Pi^{-1}\circ f_u^{-1}\circ \Pi(p).
		\end{equation}

		对每个 $q\in \mathbb{R}^2-\{0\}$,有
		\begin{equation}
			f_u^{-1}((\beta,s)) = (\dfrac 6{m_u}\cdot \beta,\dfrac{\cos(\frac{\pi}{\mu})}{\cos(\frac{\pi}6)}\cdot s).
		\end{equation}

		其中 $(\beta,s) = \Pi(q)$ 是 $q$ 的极坐标。

		令 $h:\mathbb{R}^2\rightarrow \mathbb{R}^2$ 定义为
		\begin{equation}
			h(p) = h((x,y)) = (1-x, -y)
		\end{equation}

		对每个 $p\in \mathbb{R}^2$,设其平面坐标为 $(x,y)$,
		函数 $h$ 是一个\textbf{“双反射”}:$p=(x,y)$ 依次被直线 $x=0.5$ 和 $y=0$ 反射。

		对任意两个顶点 $u,v\in \mathcal{T}$,$[u,v]$ 是一条边,令
		$g_{(u,v)}: C_u-\{0\}\rightarrow g_{(u,v)}(C_u-\{0\})$
		为
		\begin{equation}
			g_{(u,v)}(p) = R_{(v,u)}^{-1}\circ g_v^{-1}\circ h\circ g_u\circ R_{(u,v)}(p).
		\end{equation}
		对任意 $p\in C_u-\{0\}$。
		其中,$R_{(u,v)}$ 是绕 $(0,0)$ 的旋转变换,
		将 $[s_u(u) = u', s_u(v)]$ 映射到 $[u',u'_0]$。
		同理,$R_{(u,v)}^{-1}$ 也是绕 $(0,0)$ 的旋转变换,
		将 $[s_v(v) = v', v'_0]$ 映射到 $[v',v'_j]$,其中 $j$ 满足 $s_v(u) = v'_j$。

		函数 $g_{(u,v)}$ 的直观解释如下:
		它将 $C_u$ 的扇形区域 $A$ 的透镜形子集映射到 $C_v$ 的扇形区域 $B$ 的透镜形子集中。
		$A$ 和 $B$ 关系如下:
		设 $w$ 和 $z$ 满足 $[u,v,w]$ 和 $[u,v,z]$ 是 $\mathcal{T}$ 中公共边为 $[u,v]$ 的两个三角形。
		则扇形 $A$ 是 $C_u$ 包含四边形 $[s_u(u)=u',s_u(w),s_u(v),s_u(z)]$ 的扇形区域,
		扇形 $B$ 是 $C_v$ 包含四边形 $[s_v(v)=v',s_v(z),s_v(u),s_v(w)]$ 的扇形区域。
		函数 $g_{(u,v)}$ 同样是一个双射,且其逆 $g_{(u,v)}^{-1}$ 等于 $g_{(v,u)}$。
		
		$g_{(u,v)}$ 在粘连区域和转换函数的定义中有重要作用。

		\begin{Def}\label{GD-def}
			对任意 $u,v\in I$,定义粘合区域 $\Omega_{uv}$ 如下:
			\begin{equation}
				\Omega_{uv} = 
				\begin{aligned}
					\left\{
						\begin{array}{lr}
							\Omega_u, & u = v, \\
							g_{v,u}(\Omega_v)\cap \Omega_u, & [u,v]\in \mathcal{T}, \\
							\emptyset, & \rm{otherwise}.
						\end{array}
					\right.
				\end{aligned}
			\end{equation}
		\end{Def}
		我们可以证明上面的定义是满足定义 \ref{PGM-def} 中的条件 2 的。
		特别地,$\Omega_{uv} = \emptyset$ 当且仅当 $\Omega_{vu} = \emptyset$ 这个事实
		对以一致的方式定义转换函数至关重要。
		下面我们给出转换函数的形式定义:
		
		\begin{Def}\label{TF-def}
			设 $K$ 是指标集
			\begin{equation}
				K = \{(u,v)\in I\times I|\Omega_{uv}\neq 0\}.
			\end{equation}
			则对任意 $(u,v)\in K$,转换函数 $\phi_{vu}:\Omega_{uv}\rightarrow \Omega_{vu}$ 定义为
			\begin{equation}
				\phi_{vu}(p) = 
				\begin{aligned}
					\left\{
						\begin{array}{lr}
							g_{(u,v)}(p), & u\neq v, \\
							p, & u = v.
						\end{array}
					\right.
				\end{aligned}
			\end{equation}
		\end{Def}

		转换函数是双射且 $C^{\infty}$ 连续的,
		因为 $g_{(u,v)}$ 是 $C^\infty$ 连续双射函数的复合。
	
	\subsection{参数化函数的构造}
		设 $\mathcal{G}$ 是由三角网格 $\mathcal{T}$ 构造的粘连集。
		我们的目标是由 $\mathcal{G}$ 定义一族参数化函数 $\{\theta_u\}_{u\in I}$。
		为此,我们设 $|\mathcal{T}|$ 的近似为 $S'\subset \mathbb{R}^3$。
		更具体地,我们假设 $S'$ 是有限多个参数曲面片之并,即
		$S' = \cup_{\sigma\in \mathcal{T}}b_{\sigma}(\triangle),b_{\sigma}: \mathbb{R}^2 \rightarrow \mathbb{R}^3$,
		每个曲面片都对应 $\mathcal{T}$ 的一个三角形 $\sigma$ 且每个 $b_\sigma$ 的定义域都相同:$\triangle \subset \mathbb{R}^2$。
		另外,我们要求 $S'$ 至少是 $C^0$ 连续的。
		$S'$ 应通过参数化函数来描述。
		为定义参数化函数 $\theta_u$,我们定义了一族形状函数 $\{\psi_u\}_{u\in I}$ 和一族权重函数 $\{\gamma_u\}_{u\in I}$。
		
		\begin{Def}\label{SF-def}
			对 $u\in I$,我们定义形状函数 $psi_u:\square_u\subset \mathbb{R}^2\rightarrow \mathbb{R}^3$
			表示区域 $\Omega_u$ 对应的 $(m,n)$ 阶贝塞尔曲面片:
			\begin{equation}
				\psi_u(p) = \sum_{j=0}^m\sum_{k=0}^n b_{j,k}^u B_j^m(x) B_k^n(y),
			\end{equation}
			其中 $\square_u = [-L,L]^2,L=\cos(\dfrac{\pi}{m_u})$,
			$(x,y)$ 是 $p\in \square_u$ 的平面坐标,
			$\{b_{j,k}^u\in \mathbb{R}^3\}$ 是控制点,
			\begin{equation}
				B_i^l(t)=\dbinom{l}{i}\left(\dfrac{L-t}{2L}\right)^{l-i}\left(\dfrac{L+t}{2L}\right)^i
			\end{equation}
			是区间 $[-L,L]\subset \mathbb{R}$ 上的第 $i$ 个 $l$ 阶伯恩斯坦多项式。
			称 $\psi_u$ 的阶数为 $(m_u+1,m_u+1)$。
		\end{Def}

		我们通过求解最小二乘问题来确定控制点 $\{b_{j,k}^u\}$。
		事实上,$\{b_{j,k}^u\}$ 应该唯一确定一个贝塞尔曲面,
		使得它是集合 $P$ 的最小二乘意义下的最佳逼近。
		$P$ 是一个点对集合 $\{q,p|q\in P_u,p\in S'\}$,
		我们迭代计算 $P$。即
		
		\begin{enumerate}
			\item 初始 $P=\emptyset$。
			\item
				对 $\psi_u$ 的定义域 $\square_u = [-L,L]^2$ 均匀采样 $4(m_u+1)$ 个点得到集合 $Q\subset P_u$。
				注意 $\square_u$ 是最小的包含 $\Omega_u$ 的四边形,
				还要注意 $\square_u$ 均匀采样得到的点将包含不在 $P_u$ 中的点。
				这些点将不放入 $Q$ 中。
			\item
				对每个 $q\in Q$,我们找到三角形 $\tau \in \mathcal{T}$ 使得
				$q$ 包含于三角形 $s_u(\tau)\in T_u$ 中。
			\item
				计算 $q$ 相对于 $s_u(\tau)$ 的重心坐标 $(\lambda,\nu,\eta)$,
				令 $r=\lambda a+\nu b+\eta c\in \triangle=[a,b,c]$,
				$\triangle$ 是本节开头定义的 $b_{\sigma}$ 的公共定义域。
			\item
				计算 $p=b_{\tau}(r)$,将点对 $(q,p)$ 加入 $P$ 中。
		\end{enumerate}

		当 $P$ 被计算出之后,我们使用标准最小二乘算法计算 $\{b_{j,k}^u\}$ 来拟合曲面。
		
		下面我们定义权重函数 $\gamma_u$。
		我们首先定义一个数值函数 $xi:\mathbb{R}\rightarrow \mathbb{R}$:
		\begin{equation}
			\xi(t)=
			\begin{aligned}
				\left\{
					\begin{array}{lr}
						1, & t\leq H_1, \\
						0, & t\geq H_2, \\
						\dfrac 1{1+e^{2s}}, & \rm{otherwise}\\
					\end{array}
				\right.
			\end{aligned}
		\end{equation}
		其中 $H_1,H_2$ 是常数,$0<H_1<H_2<1$,
		$s = \dfrac{1}{\sqrt{1-H}} - \dfrac{1}{\sqrt H},H = \dfrac{t-H_1}{H_2-H_1}$
		$\xi(t)$ 在 $t\leq H_1$ 和 $t\geq H_2$ 时是常值,在 $H_1<t<H_2$ 时严格递减,且是 $C^{\infty}$ 的内部。
		它的任意阶导数均在 $t\leq H_1$ 和 $t\geq H_2$ 时为零,在 $t\in (H_1,H_2)$ 时非零。

		\begin{Def}\label{WF-def}
			对任意 $u\in I$,定义
			参数区域 $\Omega_u$ 对应的权重函数 $\gamma_u:\mathbb{R}^2\rightarrow \mathbb{R}$ 为
			$\gamma_u(p) = \xi(\sqrt{x^2+y^2}),p=(x,y)\in \mathbb{R}^2$
			通常我们取 $H_1=0.25H_2,H_2=\cos\dfrac{\pi}{m_u}$。
		\end{Def}

\end{document}